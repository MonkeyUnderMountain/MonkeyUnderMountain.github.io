% 编译建议:使用 XeLaTeX
\documentclass[12pt]{article}
\usepackage{ctex}
\usepackage[unicode]{hyperref}
\usepackage{geometry}
\geometry{a4paper,margin=2.5cm}
\usepackage{fontspec}
\pagestyle{plain}
\title{孤驴}
\author{}
\date{初稿于高中时代,2026年1月转为\LaTeX}

\begin{document}
\maketitle

% \begin{abstract}
% 这是一篇中文短篇小说模板,篇幅约一万字左右。内容包含引子、数个章节与结语,适合在此基础上扩展或直接修改为独立小说。小说以第一人称与第三人称交替叙述,描写一头驴在山下小村与人和自然之间发生的细微故事,主题包含记忆、回归与温柔的孤独。
% \end{abstract}

神州,奭,逐鹿时期。

虞国示县,太岳山脉某支脉。

天已昏黑,一个一身风尘,公子模样的青年走到山脚下一个屋子前,敲了敲门。山脚下只有这一个木屋,木屋旁边是一片田地,但却长满了初生的小草,是今年新荒废了的。

一个鹤发的老人开了门,显然他已经很久没有接触外人了,看到青年,有些惊讶,“公子是?”

年轻人微微一鞠躬,显得彬彬有礼,“老先生,小生项云起,自东方信国来,途经此地,不知可否留宿一晚?”说着,他掏出几枚信国的布币,“这些作为报酬。”

“好好好,这里好久没人来了。”老人说着,接过布币,将项云起引入了屋内。他找出了几个昨天剩下的窝头,放进了灶台里,生火烧水热着。不一会儿,水烧开了,老人把窝头拿出来放在一张矮桌上,又拿起了一个坛子,找了两个碗碟,也放在桌子上。

“公子莫要嫌弃,山野之地,只有这些糙饭浑酒了。”

项云起应了一句,借着还未熄灭的灶火打量着屋子。屋子不大,略显破败,灶台在东北角。东面的墙上靠着一排铁制的农具,北面则是许多缸瓮和空的粗布口袋,应该是以前用来放粮食的。屋中间就放着这张桌子,桌上放着一个铁制的摆件,是一只猛禽,不甚精美,但在这简陋的屋子里显得很不协调。

老人找出最后一坛土酒给他倒上,却看到他拿起了桌上的摆件,便问:“公子看这鸟做得如何?”

“这是老先生做的?”项云起看到了摆件底座上的一个印记,有些吃惊地问。那是一个类似月牙的凹陷,不过项云起知道,这应该是一个驴蹄印,他认识这个印记。不过马上他转念一想,这和那个人的水平相差甚远,应该是仿制品了,可这印记,却似乎丝毫不差。

“我哪有这手艺,这是我年少时一个朋友铸的。”老人又给自己倒了点酒,盘膝直接坐到了地上。他刚坐下一抬头,却看见项云起拿起了随身的佩剑。

“公子要干什么?”老人疑惑地问。

“没什么,”项云起收起了剑,心中却是无比惊讶,摆件上的印记,真的是与那个人铸出的一般无二。他压住心中的惊讶,看向老人,“先生可否给我讲下这位朋友的故事?小生认识一个人,也是一个铁匠。”

“哈哈,公子这话就不对了,这天下的铁匠何其之多?怕不是比我这荒山上的野草还多。不过,”老人拿起粗瓷碗将土酒一饮而尽,顿了顿,“公子若愿意听,我也正想啰嗦两句。山下的那群人,估计早就把他给忘了吧。”他又倒上一碗,开始了讲述。


\vspace{1em}
\hrule
\vspace{1em}

我来到这山上一个人生活,已经四十多年了吧。从我上山之时往前推十年,他还生活在示县,是个与众不同的少年。我那时也是个少年,我们那时是最好的朋友。

他叫范吕,是乡里范铁匠的儿子,我们都叫他驴子。范铁匠祖上并不是示县人。听驴子和我的父母说,范铁匠的父亲,也就是驴子的爷爷,生活在翟。他们是翟的一户普通人家。可驴子的爷爷却有个坏习性,好赌。他还有个坏脾气,犟,有大户和他耍老千,他总要钻那个牛角尖。驴子也很犟,他说这是他爷爷遗传给他的。本来不算的坏的生活,让他爷爷一次次的赌搏给败了,最后听说是因为和翟的一个大户赌输了钱,让人给活生生打死了。范铁匠那时十五岁,母亲改嫁,他被赶出了家门。

据说范铁匠一个人在虞国很多地方流浪过,学过各种手艺,靠着打铁的技术来到了示县,当了一个铁匠。这些都是很久以前的事了,那时候我还没出生哩。

我认识驴子的时候,驴子十岁,范铁匠已经四十岁了。他家里的生活很不好,全靠范铁匠一个人支持。驴子是范铁匠的长子,七岁时他才有了一个弟弟。范铁匠很宠驴子,他希望驴子有朝一日能回到翟去。而在翟孩子们都是读书的,所以纵然是生活困难也送他来私塾读了两年书,我们就是那是认识的。

驴子是个极聪明的人,和他一起在私塾时,他识字看书比我们都快。夫子很赏识他,说他是一目十行,过目不忘。有大家来讲学的时候夫子也常常带着他一起去,我父亲是乡绅,所以我也会去,我在那儿坐着一点意思也没,驴子却在一边听得津津有味。

十二岁离开私塾后,范铁匠就教驴子打铁。驴子十七岁的时候,他就能和他爸一起打铁了。

不过很快,驴子就和范铁匠发生了冲突。范铁匠去了一遭翟,找了很多朋友,帮驴子联系到了一个铁匠师傅。他回到示,才告诉驴子,可驴子却并不想去。他和范铁匠大吵了一场,跑出了家,来找我喝酒。

驴子性格很怪,在乡里也只有我一个朋友。范铁匠喜欢喝酒,他也染上了这个习惯,总是喜欢来找我喝酒。他那天喝得大醉,说着一些不着边际的话。他说,如今是大乱之世,各国都在招募人才,积蓄实力,连这小县里的两个小家族曹家和魏家都在较劲。只要是有才之士,就可以有一番作为,而不像奭初必须是贵族才能建功立业。虞国如今虽有实力,却偏安一隅,国君只想保住前辈的基业,却不想一展宏图,成就霸业。东方的信国,西方的涿国,北方的平国,南方的襄国,那才是真正的梧桐,凤凰应该在那里安栖。示县位于信国和涿国的通道上,这里能结交四方有志之士,而翟却深居虞国腹地,没有多少人来投奔虞君,去了那里,真的就只能打一辈子的铁了。他的人生不应该是那样的,有朝一日他会离开示,离开虞,去神州上闯出一个名堂来的。我听得热血沸腾,期盼着有一天能跟着他去广阔的神州九野闯荡一番,只是可惜……


\vspace{1em}
\hrule
\vspace{1em}

说到这里,老人又猛地灌下一碗酒,这已经是第三碗了。而他手边的窝头,还一点没动。项云起有些担心老人,老人却只是把手一挥,“无妨,五十多年过去了,这故事也早被人们给忘了吧,只剩我这个将死之人抱着它在这老林里一人生活。今天幸得公子这么好的听客,我讲讲也高兴。”接着,老人又继续着他的讲述。


\vspace{1em}
\hrule
\vspace{1em}

驴子和范铁匠争执了半个月,最后范铁匠妥协了。驴子还是继续打铁,偶尔有人来讲学他也会去县里听。我家里有几本父亲的藏书,全都被他借去抄了一遍。范铁匠越来越后悔送他读书识字,也十分反感他读书,但他依然我行我素,和范铁匠之间的隔阂也越来越大。

他有次找我来喝酒,和我一起上了村边的土坡。那时已是黄昏,比田地还要大的红红的太阳就挂在西山上。

他说,赵安,你看这太阳,多美,多让人向往。可是马上它就要落下山去了,太阳落山了世界就会变得一片黑暗。没人喜欢黑暗,却很少有人去追寻光明。

然后他给我讲了个故事,他说上古有一个英雄,是叫夸父的。夸父是一个小部落里的一个普通人,没有人关心他,他只能从太阳的光里得到一些温暖,可当黑夜降临时,连太阳也会离他而去。所以他讨厌黑暗,他想要追上太阳的脚步,把它抓住,栓在部落的天上。部落里的人都说他疯了,也没有人管他。于是他只好独自一个人出发。夸父攒够了钱,找到巫师,用他的生命为代价在自己的身体上绘上了代表巨大的山海纹,在他的手杖上绘上了代表风的五行纹,这样他就可以像山一样高大,奔跑起来像风一样快了。他追啊追,可太阳传说中是大妖三足乌的化身,纵然夸父拼上了生命,绘出了绝世的道纹,也没有大妖的力量强大。不过跑了几步,夸父的生命被道纹榨干,轰然倒下了。

驴子说,这个故事是他从书上读来的。刚读完的时候他觉得夸父好傻,为一件完全没必要的事搭上自己的性命。可是再读时他却觉得夸父是对的。夸父的生活没有任何快乐,只有太阳可以给他温暖,所以他要追上太阳,就像飞蛾扑向火焰,在那一刻他们一定是快乐的。

然后他说,他想离开示去寻找属于自己的太阳。我当时和他击掌为誓,说不管去哪里,我都会陪着他的。


\vspace{1em}
\hrule
\vspace{1em}

“可是不过一年,我就违背了这誓言。”老人把碗中的残酒饮完,幽幽长叹一声。

些时天已经完全黑了,灶中的火也已熄灭。老人摸索出一段陈年的旧烛,放在桌上点燃,又坐在地上,“公子莫要嫌我啰嗦,这故事虽不怎么有趣,却也不该被遗忘。”

“老先生请继续,在下对此很感兴趣,只怕老先生不讲,在下还不愿呢。”项云起仍是文质彬彬地说。他不知道,那人名扬神州,年少时还有这样的孤独。


\vspace{1em}
\hrule
\vspace{1em}

我记得很清楚,虞宪公七年的夏天。那年夏,示县出奇地多雨。一个下午,驴子正在我家和我饮酒。明明太阳还好好地挂在天上,突然一个霹雳下来,那天就像关上了窗子一样,刹地就黑了,接着雨点便打了下来,噼里啪啦地,就跟天上往下倒豆子似的。

驴子和我刚放下酒杯,便有人急切地敲门。驴子出去开了门,引进来一个侠客模样的人。后来驴子说,那人是侠家的一位野牧,他认识那人剑上的佩饰,是侠家的鸱鹞衔云佩。我也看到了那枚佩,不过我不知道野牧是什么,但就觉得那人气宇轩昂,很是不凡。

他说他叫郭承,从信国来,要往涿国去。驴子和他谈论,他也只是不断打趣,并像那些来讲学的人张口闭口家国兴亡。但他无意间谈到,信国那年遭了天灾,几场冰雹让农田颗粒无收。这让驴子看到了机会。

很快天晴了,郭承便走了。驴子也马上就回家了。第二天,驴子来找我,说要贩粮去信国。我很惊讶,但也决定跟随他。驴子那两天不断打探消息,确认了信国遭到天灾,随后便张罗起来。

当时最大的问题是没有本钱。我不敢和父母说要去贩粮,他们是乡绅,看不起这种事。那时是七月,九月份粮食就收获了,只有两个月的时间。驴子整天苦着一张脸,找不到筹钱的方法。差不多过了半个月吧,驴子忽然想到了去找曹家公子。

曹家和魏家都是示县的大族,曹家过去是示地的大夫,后来改革了县制,曹家失去了封地,但实力还在。而魏家,则是一个新兴的家族,靠着家长是国君身边红人的红人,最近才发展起来的。但那时,那位红人有些失宠,连带着魏家家长也失了势。魏公子在家主持家务,处处受到曹家的打压,驴子说这是个好机会。

那天夜里,驴子家的铁匠铺亮了一夜,第二天驴子就秘密地带我去拜访曹公子,范铁匠极为反感这些世家大族,驴子不敢让他知道。刚开始,魏家的家奴不让我们进去,他就给家奴递上了一个佩饰。你说他多聪明,那佩竟和郭承的长得极像,就凭半个月前的印象,他硬生生给仿制了一个。后来他和我说其实那个佩他做得很——

果然,很快魏公子就满脸堆笑地出迎接我们了,还一口一个公子地叫着。驴子也顺着他的意思,大步进了门内,我却慌张得不行,跟在驴子后面,就好像他的随从一样。到了大堂坐下,驴子才从容不迫地告诉魏公子那佩是假的。魏公子刚还不信,再一细看那佩,也看出了不对,立刻就勃然大怒。我吓出了一背冷汗,驴子却悠闲地饮了一口魏家的茶,才缓缓地开口和魏公子解释。那段话才叫精彩,五十多年过去了,我到现在也忘不了。

他先是站起来鞠了个躬,然后问魏公子:“公子准备如何处置我呢?”

魏公子冷笑一声,“当然是乱棍打出。”

驴子不慌不忙地说:“那公子觉得我为何要来白白挨这一顿打呢?”

魏公子好像觉得有点意思,发了个鼻音“嗯?”

驴子说:“因为我知,公子不是那蛮横无理,棍打有识之士之人。”

马上他就滔滔不绝地说了起来:“公子气宇非凡,将来是要成大事之人,如今却是龙游浅滩,空有一身本领却施展不得,对吧?在示县这小地方,还要处处被曹家这虾给压着,公子可觉气愤?这曹家虽说以前是示的大夫,可如今也不过是县里的平民而已,为何他们就能在这示县横着走,连你这朝中官员之子也要让着他呢,公子可曾想过?

“不知公子是否想过,鄙人却替公子十分气愤,时常思想这事。想来想去,不过就是一点,他曹家就胜过了公子。公子可知是何?

“不过只是他曹家有祖上传下来的田地,比公子的多而已,也就是说,他曹家,比公子有钱。

“小人此次冒着这一顿乱棍来见公子,正是为公子献这生财之道。”

魏公子听完这番话,马上来了兴致,挥手让家仆出去,只留下我们三人在堂中。他请驴子坐下,问:“先生这生财之道是?”

驴子一点也不客气,坐下又喝了口茶说:“公子可听说信国的灾情?”

“略有耳闻。”

“这农田遭灾,收成必然不好。此时若公子运千石粮至信国,以倍价售出,信国少粮,虽然是高价也可以卖出到那些大户手中,然后换成布匹运回示县。如此一次往返,不过两月。而公子纵有百顷良田,春耕夏耘,无数佣工忙活一年,只一支车队,两月时间便可翻倍。公子意下如何?”

魏公子本来很期待的脸这时皱起了眉头,说:“行商之事,偷巧取利,为人所不齿。”

我一听这话,一下就感觉完了。可驴子还是毫不慌张地继续说:“我给公子讲个故事吧。

“以前妖界是有个大妖帝江,统御一方妖众的。这帝江,本名唤作浑沌,长得就像公子家中饲养的猪豚一样。传说他与那肥猪,本是一族。他刚出生的那个时代,我们人族的大帝们都还没有出生,妖界十分混乱,各种大妖环视。猪这一族,本也是长着獠牙的凶兽。可这猪呢,偏偏和一般妖想得不一样,它总觉得自己太凶残了,于是打自己的獠牙打断了。很快,它们就变成了其它妖族的食物,所以我们如今常说蠢猪。只有帝江没有这样,他靠着自己的獠牙和凶狠,吃掉其它的妖,最后成为了大妖帝江。如今这机会,不正是天赐给公子的獠牙吗?公子若是放弃这机会,不就像猪放弃它天生的獠牙一样吗?是自断獠牙任人宰割,还是利用好它,做统御一方的大妖帝江,全看公子决断了!”

那魏公子听完这段话,先是一愣,随即便哈哈大笑,说:“我还是想做那大妖帝江的啊。”

说完,便留下我们喝酒,住了一晚。


\vspace{1em}
\hrule
\vspace{1em}

讲到这段的时候,老人直接站了起来,在烛火的映照下模仿着驴子和魏公子的一举一动。他已经喝下了半坛酒,脸色发红,说到驴子喝茶时便喝一口酒,说到魏公子大笑时竟真的大笑起来。

他因苍老而浑浊的眼珠在这时也变得得明亮起来,烛光照了进去,在里面闪烁。


\vspace{1em}
\hrule
\vspace{1em}

第二天我和驴子离开了曹家,驴子那么高兴的时候我这一辈子也就见过那一次。

只是,唉。

还没走出十里路,魏家就出事了。魏家在宫里的靠山倒了,被满门抄斩,曹家比魏家先得到消息,马上联系了县丞,给魏家安上罪名,全家打入狱中。我和驴子还路上兴高采烈地谈论着贩粮的细节,衙役便已骑马赶到,把我们也投进牢中。

这事对驴子打击极大,当时他差点昏过去。我俩在牢里待了一个月,最后靠我父亲给县丞送礼才出来。那一个月,每天都被拉去当苦力,驴子一个月几乎一句没说。县役看他不爽骂他,他就瞪县役,因为这,他挨了不少打。

出狱的时候只有我父母来了,范铁匠和驴子的母亲都没来。他们看见我就抱上我哭,驴子自已一个人站在一边,面无表情,还是一句话也不说。

等到父亲情绪平静了,才走过去和驴子说,范铁匠让驴子自己回去,他有话和驴子说。驴子听完,朝我挥了挥手,便一言不发地走了。父亲是架着马车来的,他让我和母亲上车,架车从驴子身边驰过。我和父亲说,让驴子也上来吧。父亲马上就瞪了我一眼,脸上的凶狠我之前从没有见过。

“范铁匠说了,让他自己走回去!你以后不许再和他接触!”他当时就这样训斥我。

我刚想说什么,他马上又回过头来瞪了我一眼,“闭嘴!”

我不敢再说话了,只能回头看看驴子。驴子一个人走在大路上,身上还穿着囚服,也不跑,也不停下,只是一直地向前走。可他离我却越来远,不一会儿就消失在我的视线之外。

后来听说,范铁匠那天把驴子抄录的书藉全搬到了道边,就在那儿守着驴子。等驴子回来时,当着他的面把那些书全烧了。驴子惊呆了,但还是在一边什么话也不说。

那天驴子没有走范铁匠的家门。他在范铁匠门前跪了一夜。范铁匠刚开始也莽,就让驴子跪着去了,后来等他心软时让驴子回家,那头犟驴还不回去。等到天亮时,他冲家里喊,说之后范铁匠再无他这个不孝子,喊完就头也不回地走了,邻居们有人好心拉他,都被他甩开了。

那天下午,他忽然找到我。我当时一个人在家喝闷酒,他就推门进来了。可笑的是,我当时心里不是惊喜,竟然是想我父母看到没。

我给他倒了一杯酒,他也不客气,拿起来就干了。我什么话也没说,只是不停地喝酒,给他倒酒。他干了第三杯的时候忽然抬起头来看着我,我也有些惊讶地看着他,听他说出那些话。

他说,他要再去找曹家公子。

我觉得他疯了,在牢里那一个月还不够吗?我质问他。

他笑笑,摇了摇头,和我说,“赵安,我不甘心,我真的不甘心,就在这乡野之中打一辈子铁。上天让我生在了这个时代,又给我这个机会,我不会就这样放弃的。

“我们在牢里待了一个月,在牢里受了一个月的苦。可我们什么也没有做错啊!凤凰浴火而生,如果他在火焰中退缩了,便只能做一只普通的乌雀,那它在火焰中受的那些煎熬也是白受了。我此时退缩,这一个月的苦不是白受了吗?

“所以,赵安,你愿意和我一起去吗?”

他有些疲惫地看着我,眼神中的那一种期待,我到现在也记得。

可我当时真是个懦夫!我对他吼,说:“你再走进那火焰里,会被烧死啊!”

他听完之后继续看我,他那个眼神,可能是失望吧,我不知道,也描述不出来。

“那便被烧死吧,还记得我和你说的夸父吗?他死的时候,也是满足的,因为那是,他真正想做的事啊。”过了一会儿,驴子这样说。然后他起身对我行了个礼,说:“那么乡绅大人,我去那火焰里了。”

说完,他便走了。我就看着他走了,没有挽留,也没有追上去。

他真的就那么走了,再也没有回来。


\vspace{1em}
\hrule
\vspace{1em}

老人的讲述停止了,放下了手中的酒碗看着屋外。明月已经升到天空中央,在在地上投下一片银辉。

项云起也低头不说话,回味着老人讲的故事。忽然他抬起头来问老人:“先生知道他之后的消息吗?”

老人回过神来,叹了口气,说:“他去找曹家没有成功,直接被乱棍打出来了。但是他也没回到乡里,而是一个人离开了示县。之后三五年也没有再听说过他的消息。可差不多他离开六年后吧,不知哪里来的传言,说信国如今的第一豪商就是他,他离开示之后就改名不叫范吕了,改成什么我也忘了。又过了一年,他回到示,要接范铁匠离开。范铁也犟,说与他再无瓜葛,死活还要留在示打铁。他家的犟真是遗传的,祖孙三代,一个比一个犟。后来驴子说是留下了点钱,走了。

“那之后不久,我们家便破落了,我搬到了这山上来住,偶尔下山去,也再没听到过有关他的消息。”

“老先生,我认识的那个铁匠,应该就是您说的驴子了。”项云起拿起那个摆件,又拿出自己的佩剑,上面都有一个一样的驴蹄印,“他可是喜欢在自己的作品上印出这样的印记?”

听到他的话,老人一惊,马上拿起摆件和项云起的剑,先是看了一眼,又细细端详,点了点头,“嗯,驴子确实有这个习惯,他说他打出来的东西,都是独一无二的精品,应该有独一无二的印记。”老人放下摆件和剑,看向项云起,眼神中带着迫切,“那公子可知道关于驴子的消息?”

“他应该是改名叫吕子成了。子成大师四十年前从信国发迹,靠经商起家,很快就成了信国最大的豪商。随后他扶持信国公子玉登上国君之位,也就是信成公。成公上位后他做了三年上卿,而后南下去了襄国经商,两次散尽家财。之后世人便不知道他的去向。其实他回到了信国,隐居起来,成了侠家的一名野牧,专心做铁匠,冶出了很多绝世的武器。”项云起顿了一下,想起老人并不知道野牧,便又解释道:“野牧一共只有九名,是侠家地位最高的称号。”

老人脸上的表情不断变化,先是震惊,而后忽然大笑,笑着笑着还有眼泪流出。他抱起所剩不多的酒坛,因为醉酒而摇摇晃晃地走出屋去,对着明月大喊:

“驴子啊,你成了凤凰了啊!”他拿起酒坛大饮两口,又把剩下的酒洒在地上,“赵安敬你!”

说罢,他踉跄几步,“砰”地倒在了地上。

项云起忙赶过去看,却见老人头磕在一块石头上,血从石头上流下。

老人死了。


\end{document}

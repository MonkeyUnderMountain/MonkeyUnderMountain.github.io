\section{Affinoid algebras}

\subsection{The first properties}

    \begin{definition}\label{def:affinoid_algebras}
        Let \(\kk\) be a non-archimedean field. 
        A banach \(\kk\)-algebra \(A\) is called a \emph{ affinoid \(\kk\)-algebra} if there exists an admissible surjective homomorphism
        \[
            \varphi: \kk\{ \underline{T/r} \} \twoheadrightarrow A
        \]
        for some \(r = (r_1, \ldots, r_n) \in \bbR_{>0}^n\).

        If one can choose \(r_1 = \cdots = r_n = 1\), then we say that \(A\) is a \emph{strict affinoid \(\kk\)-algebra}.
    \end{definition}

    \begin{definition}\label{def:restricted_Laurent_series}
        Let \(\kk\) be a non-archimedean field.
        We define the \emph{ring of restricted Laurent series} over \(\kk\) as 
        \[ \KK_r = \bfL_{\kk,r} = \left\{\sum_{n \in \bbZ} a_n T^n : a_n \in \kk, \lim_{|n| \to \infty} |a_n| r^n = 0 \right\} \]
        equipped with the norm
        \[ \|f\| = \sup_{n \in \bbZ} |a_n| r^n. \]
    \end{definition}

    \Yang{Is \(\KK_r\) always a field?}
    \Yang{Do we have \(\bfL_{\kk,r} = \Frac (\kk\{T/r\})\)?}

    \begin{proposition}\label{prop:restricted_Laurent_series_is_a_field_when_r_is_not_root_of_absolute_value}
        Let \(\kk\) be a non-archimedean field.
        If \(r \notin \sqrt{|\kk^\times|}\), then \(\KK_r\) is a complete non-archimedean field with non-trivial absolute value extending that of \(\kk\).
    \end{proposition}

    \begin{proposition}\label{prop:affinoid_algebra_is_noetherian_and_ideal_is_clsoed}
        Let \(A\) be an affinoid \(\kk\)-algebra.
        Then \(A\) is noetherian, and every ideal of \(A\) is closed.
    \end{proposition}
    \begin{proof}
        \Yang{To be completed.}
    \end{proof}

    \begin{proposition}\label{prop:the_norm_on_affinoid_is_bounded_by_spectral_radius}
        Let \(A\) be an affinoid \(\kk\)-algebra.
        Then there exists a constant \(C > 0\) and \(N > 0\) such that for all \(f \in A\) and \(n \geq N\), we have
        \[
            \|f^n\| \leq C \rho(f)^n.   
        \]
    \end{proposition}
    \begin{proof}
        \Yang{To be completed.}
    \end{proof}

    \begin{proposition}\label{prop:affinoid_algebra_is_strict_when_radius_in_the_radical_of_valuation_set}
        Let \(A\) be an affinoid \(\kk\)-algebra.
        If and only if \(\rho(f) \in \sqrt{|\kk|}\) for all \(f \in A\), then \(A\) is strict.
        \Yang{To be complete.}
    \end{proposition}
    \begin{proof}
        \Yang{To be completed.}
    \end{proof}

\subsection{Noetherian normalization theorem}

    \begin{theorem}\label{thm:noetherian_normalization_theorem}
        Let \(A\) be an affinoid \(\kk\)-algebra.
        Then there exists a finite injective homomorphism
        \[
            \varphi: \kk\{ r_1^{-1} T_1, \ldots, r_d^{-1} T_d \} \hookrightarrow A
        \]
        for some \(d \in \bbN\) and \(r_1, \ldots, r_d \in \bbR_{>0}\).
        \Yang{To be checked.}
    \end{theorem}


\subsection{Tate algebras and Weierstrass division}

    \begin{definition}\label{def:distinguished_degree_of_tate_algebra}
        Let \(R\) be a non-archimedean banach ring and \(r \in \bbR_{>0}\).
        A restricted power series \(f = \sum_{\alpha \in \bbN^n} a_\alpha T^\alpha \in R\{ \underline{T/r} \} \) is said to be \emph{distinguished in the variable \(T_n\) of degree \(d\)} if
        \begin{itemize}
            \item \(a_{\alpha} \in R\) is a unit for \(\alpha = (0, \ldots, 0, d)\);
            \item \(\|a_\alpha\| r^\alpha < \|a_{(0,\ldots,0,d)}\| r_n^d\) for all \(\alpha_n < d\).
        \end{itemize}
        \Yang{To be revised.}
    \end{definition}

    \begin{proposition}\label{prop:restricted_power_series_invertible_iff_the_constant_item_is_invertible_and_controls_others}
        Let \(R\) be a non-archimedean banach ring.
        An element \(f = \sum_{\alpha \in \bbN^n} a_\alpha T^\alpha \in R\{ \underline{T/r} \} \) is invertible if and only if \(a_0\) is invertible in \(R\) and \(\|a_0\| > \|a_\alpha\| r^\alpha\) for all \(\alpha \neq 0\).
        % \Yang{To be checked.}
    \end{proposition}
    \begin{proof}
        Multiplying by \(a_0^{-1}\), we can reduce to the case \(a_0 = 1\).
        Let \(g = \sum_{\alpha \in \bbN^n} b_\alpha T^\alpha\) be the inverse of \(f\) in \(R[[\underline{T}]]\).
        Then we have
        \[ f \cdot g = \sum_{\alpha \in \bbN^n} a_\alpha T^\alpha \cdot \sum_{\beta \in \bbN^n} b_\beta T^\beta = \sum_{\gamma \in \bbN^n} \left( \sum_{\alpha + \beta = \gamma} a_\alpha b_\beta \right) T^\gamma = 1. \]
        That is, for every \(\gamma \neq 0 \in \bbN^n\),
        \[ b_{\gamma} = - \sum_{\substack{\alpha + \beta = \gamma \\ \alpha \neq 0}} a_\alpha b_\beta. \]
        Let \(A = \|f-1\| < 1\).
        We show that for every \(m \in \bbN\), there exists \(C_m > 0\) such that for all \(\alpha \in \bbN^n\) with \(|\alpha| \geq C_m\), we have \(\|b_\alpha\| r^\alpha \leq A^m\).
        For \(m = 0\), note that \(b_0 = 1\).
        By induction on \(\gamma\) with respect to the total order \(\leq_{\text{total}}\), we have
        \[ \|b_\gamma\| r^\gamma \leq \max_{\substack{\alpha + \beta = \gamma \\ \alpha \neq 0}} \|a_\alpha\| r^\alpha \cdot \|b_\beta\| r^\beta \leq A \max_{\beta <_{\text{total}} \gamma} \|b_\beta\| r^\beta \leq 1. \]
        Suppose that the claim holds for \(m\).
        There exists \(D_{m+1} \in \bbN\) such that for all \(\alpha \in \bbN^n\) with \(|\alpha| \geq D_{m+1}\), we have \(\|a_\alpha\| r^\alpha \leq A^{m+1}\).
        Set \(C_{m+1} = C_m + D_{m+1} + 1\).
        For any \(\gamma \in \bbN^n\) with \(|\gamma| \geq C_{m+1}\), we have
        \[ \|b_\gamma\| r^\gamma \leq \max_{\substack{\alpha + \beta = \gamma \\ \alpha \neq 0}} \|a_\alpha\| r^\alpha \cdot \|b_\beta\| r^\beta \leq \max \left\{ A^{m+1}, A \cdot A^m \right\} = A^{m+1} \]
        since either \(|\alpha| \geq D_{m+1}\) or \(|\beta| \geq C_m\).
        Thus by induction, we have \(\|b_\alpha\| r^\alpha \to 0\) as \(|\alpha| \to +\infty\).
        It follows that \(g \in R\{ \underline{T/r} \} \).
    \end{proof}

    \begin{theorem}[Weierstrass preparation theorem]\label{thm:Weierstrass_preparation_theorem_Tate_algebra}
        Let \(\kk\) be a complete non-archimedean field.
        Let \(f \in \kk\{ \underline{T/r} \} \) be a restricted power series that is distinguished in the variable \(T_n\) of degree \(d\), i.e.,
        \[ f = \sum_{\alpha \in \bbN^{n-1}} a_\alpha T^\alpha + \sum_{\alpha_n \geq 1} a_\alpha T^\alpha \]
        with \(a_{(0,\ldots,0,d)}\) being a unit in \(\kk\{ \underline{T/r} \} \) and \(\|a_\alpha\| r^\alpha < \|a_{(0,\ldots,0,d)}\| r_n^d\) for all \(\alpha_n < d\).
        Then there exists a unique monic polynomial \(P \in \kk\{ \underline{T/r} \} [T_n]\) of degree \(d\) in \(T_n\) and a unique unit \(U \in \kk\{ \underline{T/r} \} \) such that
        \[ f = P \cdot U. \]
        \Yang{To be checked.}
    \end{theorem}

    \begin{theorem}[Weierstrass division theorem]\label{thm:Weierstrass_division_theorem_Tate_algebra}
        Let \(\kk\) be a complete non-archimedean field.
        Let \(f \in \kk\{ \underline{T/r} \} \) be a restricted power series that is distinguished in the variable \(T_n\) of degree \(d\).
        Then for every \(g \in \kk\{ \underline{T/r} \} \), there exists a unique \(Q \in \kk\{ \underline{T/r} \} \) and a unique polynomial \(R \in \kk\{ \underline{T/r} \} [T_n]\) of degree less than \(d\) in \(T_n\) such that
        \[ g = Q \cdot f + R. \]
        \Yang{To be checked.}
    \end{theorem}

    \begin{proposition}\label{prop:algebraic_spectra_of_Tate_algebra}
        Let \(\kk\) be a complete non-archimedean field and \(r = (r_1, \ldots, r_n) \in \bbR_+^n\).
        Then 
        \[ \Spec \kk\{ \underline{T/r} \} = \{  \}, \]
        where
    \end{proposition}

% \subsection{Reduction}
 
%     \begin{definition}\label{def:reduction_ring_of_nonarchimedean_banach_ring}
%         Let \(R\) be a non-archimedean banach ring.
%         We define 
%         \[
%             R^\circ = \{ f \in R : \rho(f) \leq 1 \}, \quad R^{\circ \circ} = \{ f \in R : \rho(f) < 1 \}.
%         \]
%         The \emph{reduction} of \(R\) is defined as the quotient ring
%         \[ \widetilde{R} = R^\circ / R^{\circ \circ}. \]
%     \end{definition}

%     For a non-archimedean field \(\kk\), its reduction ring \(\widetilde{\kk} = \calk_\kk\) is just the residue field of its valuation ring.

%     \begin{example}\label{eg:reduction_ring_of_trivial_normed_rings}
%         Let \(R\) be a ring equipped with the trivial norm.
%         Then we have \(R^\circ = R\) and \(R^{\circ \circ} = \nil(R)\).
%     \end{example}

%     \begin{example}\label{eg:reduction_ring_of_tate_algebra_over_a_banach_ring}
%         Let \(R\) be a non-archimedean banach ring and \(A = R\{T\}\) be the Tate algebra in one variable over \(R\).
%         %  field and \(A = \kk\{ \underline{T} \}\) be a Tate algebra over \(\kk\).
%         Then we have
%         \[ A^\circ = \left\{ \sum_{n\geq 0} a_n T^n : |a_n| \leq 1 \text{ for all } n \in \bbN \right\}  = R^\circ \{ T \}, \]
%         and
%         \[ A^{\circ \circ} = \left\{ \sum_{n\geq 0} a_n T^n : |a_n| < 1 \text{ for all } n \in \bbN \right\}  = R^{\circ \circ} \{ T \}. \]
%         Since the norm of items in a restricted power series will tend to \(0\), we have
%         \[ \widetilde{A} = \widetilde{R} [\underline{T}]. \] 
%         % More generally, for \(r = (r_1, \ldots, r_n) \in \bbR_{>0}^n\), we have

%         More generally, for \(r \in \bbR_{>0}^n\), consider the Tate algebra \(A = R\{T/r\}\).
%         Then we have
%     \end{example}


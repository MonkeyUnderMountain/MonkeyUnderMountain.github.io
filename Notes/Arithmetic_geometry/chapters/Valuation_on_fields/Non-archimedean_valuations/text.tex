\section{Non-archimedean valuations}

\subsection{Topology: Ultra-metric space}

    We will use \(B(x,r)\) (resp. \(E(x,r)\)) to denote the open ball (resp. closed ball) with center \(x\) and radius \(r\).
    % We will use \(E(x,r)\) to denote the closed ball with center \(x\) and radius \(r\).

    \begin{definition}\label{def:ultra-metric_space}
        A metric space \((X,d)\) is called an \emph{ultra-metric space} if its metric \(d\) satisfies the \emph{strong triangle inequality}:
        \[ d(x,z) \leq \max\{d(x,y), d(y,z)\},\quad\forall x,y,z\in X. \]
    \end{definition}

    \begin{remark}\label{rmk:Chinese_translation_of_ultra-metric_space}
        The term \emph{ultra-metric space} should be translated into Chinese as ``\emph{奥特度量空间}''.
        There is no special reason for this translation, except that I insist on using ``奥特'' to translate ``ultra''.
    \end{remark}

    If \((\kk,\|\cdot\|)\) is a non-archimedean field, then the metric \(d(x,y) := \|x-y\|\) on \(\kk\) makes \((\kk,d)\) an ultra-metric space.

    \begin{proposition}\label{prop:all_triangles_in_ultra-metric_space_are_isosceles}
        Let \((X,d)\) be an ultra-metric space.
        Then for any \(x,y,z \in X\), at least two of the three distances \(d(x,y), d(y,z), d(z,x)\) are equal.
        And the third distance is less than or equal to the common value of the other two.
    \end{proposition}
    \begin{proof}
        Suppose that \(d(x,y) \geq d(y,z)\).
        By the strong triangle inequality, we have
        \[ d(x,z) \leq \max\{d(x,y), d(y,z)\} = d(x,y). \]
        On the other hand, by the strong triangle inequality again, we have
        \[ d(x,y) \leq \max\{d(x,z), d(z,y)\} = \max\{d(x,z), d(y,z)\} \leq d(x,y). \]
        This shows that \(d(x,y) = \max\{d(x,z), d(y,z)\}\).
        Thus either \(d(x,z) = d(x,y) \geq d(y,z)\) or \(d(y,z) = d(x,y) \geq d(x,z)\).
        %\Yang{To be continued.}
    \end{proof}

    \begin{proposition}\label{prop:balls_in_ultra-metric_space_form_a_tree}
        Let \((X,d)\) be an ultra-metric space.
        Let \(D_i\) be (open or closed) ball in \(X\) for \(i=1,2\).
        If \(D_1 \cap D_2 \neq \emptyset\), then either \(D_1 \subseteq D_2\) or \(D_2 \subseteq D_1\).
    \end{proposition}
    \begin{proof}
        Suppose that \(D_i\) has center \(x_i\) and radius \(r_i\) for \(i=1,2\).
        Let \(y \in D_1 \cap D_2\).
        We have 
        \[ d(x_1,x_2) \leq \max\{d(x_1,y), d(y,x_2)\}. \]
        Without loss of generality, we may assume that \(d(x_1,x_2) \leq d(x_1,y)\).
        It follows that \(x_2 \in D_1\) since \(d(x_1,y) < r_1\) (or \(\leq r_1\)).

        If there exists \(z \in D_2 \setminus D_1\), we claim that \(D_1 \subseteq D_2\).
        We have \(d(x_1,z) > d(x_1,x_2)\).
        Then by \cref{prop:all_triangles_in_ultra-metric_space_are_isosceles},
        \[ r_1 \leq d(x_1,z) = d(x_2,z) \leq r_2. \]
        In particular, if \(D_2\) is an open ball, then we have strict inequality \(r_1 < r_2\).
        For any \(w \in D_1\), we have 
        \[ d(x_2,w) \leq \max\{d(x_2,x_1), d(x_1,w)\} \leq r_1 \leq r_2. \]
        Thus \(w \in D_2\) whatever \(D_2\) is open or closed, and it shows that \(D_1 \subseteq D_2\).
    \end{proof}

    \begin{proposition}\label{prop:all_balls_in_ultra-metric_space_are_clopen}
        Let \((X,d)\) be an ultra-metric space.
        Then both \(B(x,r)\) and \(E(x,r)\) are closed and open subsets of \(X\) for any \(x \in X\) and \(r > 0\).
    \end{proposition}
    \begin{proof}
        We show that the sphere \(S(x,r) := \{y \in X \mid d(x,y) = r\}\) is open in \(X\).
        Note that if \(y \in S(x,r)\), then for any \(r' < r\), we have \(B(y,r') \cap E(x,r) \neq \emptyset\) and \(x \in E(x,r) \setminus B(y,r')\).
        Thus by \cref{prop:balls_in_ultra-metric_space_form_a_tree}, we have \(B(y,r') \subseteq E(x,r)\).
        If \(B(y,r') \cap B(x,r) \neq \emptyset\), then by \cref{prop:balls_in_ultra-metric_space_form_a_tree} again, we have \(B(y,r') \subseteq B(x,r)\).
        However, \(y \in B(y,r') \setminus B(x,r)\), a contradiction.
        Thus \(B(y,r') \subseteq E(x,r) \setminus B(x,r) = S(x,r)\).
        It yields that \(S(x,r) = \bigcup_{y \in S(x,r)} B(y,r/2)\) is open in \(X\).

        Since \(E(x,r) = B(x,r) \cup S(x,r)\) and \(B(x,r) = E(x,r) \setminus S(x,r)\), both \(B(x,r)\) and \(E(x,r)\) are open and closed in \(X\).
    \end{proof}

    \begin{corollary}\label{prop:ultra-metric_space_is_totally_disconnected}
        Let \((X,d)\) be an ultra-metric space.
        Then \(X\) is totally disconnected, i.e., the only connected subsets of \(X\) are the set with at most one point.
    \end{corollary}
    \begin{proof}
        Suppose that \(S \subset X\) has at least two distinct points \(x,y \in S\).
        Let \(r := d(x,y) > 0\).
        Consider the open ball \(B(x,r/2)\).
        By \cref{prop:all_balls_in_ultra-metric_space_are_clopen}, \(B(x,r/2)\) is both open and closed in \(X\).
        Thus \(B(x,r/2) \cap S\) is both open and closed in \(S\), however, it is non-empty and not equal to \(S\) since it contains \(x\) but not \(y\).
        This shows that \(S\) is disconnected.
    \end{proof}

    \begin{proposition}\label{prop:convergence_in_ultra-metric_space}
        Let \((X,d)\) be an ultra-metric space.
        A sequence \(\{x_n\}\) in \(X\) is cauchy if and only if \(d(x_n,x_{n+1}) \to 0\) as \(n \to \infty\).
    \end{proposition}
    \begin{proof}
        The necessity is true for all metric spaces.
        Suppose that \(d(x_n,x) \to 0\) as \(n \to \infty\).
        For any \(\varepsilon > 0\), there exists \(N \in \bbN\) such that \(d(x_n,x_{n+1}) < \varepsilon\) for all \(n \geq N\).
        For any \(m,n \geq N\) with \(m < n\), by the strong triangle inequality, we have
        \[ d(x_n,x_m) \leq \max\{d(x_n,x_{n-1}), d(x_{n-1},x_m)\} \leq \max\{d(x_n,x_{n-1}), d(x_{n-1},x_{n-2}), \ldots, d(x_{m+1},x_m)\} < \varepsilon. \]
        This shows that \(\{x_n\}\) is a cauchy sequence.
    \end{proof}


\subsection{Algebra: ring of integers and residue field}

    Let \(\kk\) be a non-archimedean field.
    Then easily see that \(\{x \in \kk\colon \|x\|\leq 1\}\) is a subring of \(\kk\).
    Moreover, it is a local ring whose maximal ideal is \(\{x \in \kk\colon \|x\| < 1\}\).

    \begin{definition}\label{def:non-archimedean_field_ring_of_integers_maximal_ideal_and_residue_field}
        Let \(\kk\) be a non-archimedean field.
        The \emph{ring of integers} of \(\kk\) is defined as
        \[ \kk^\circ := \{x \in \kk\colon \|x\|\leq 1\}. \]
        Its maximal ideal is
        \[ \kk^{\circ\circ} := \{x \in \kk\colon \|x\| < 1\}. \]
        The \emph{residue field} of \(\kk\) is defined as
        \[ \calk_\kk := \widetilde{\kk} := \kk^\circ / \kk^{\circ\circ}. \]
    \end{definition}

    % In this subsection, let \(\kk\) be a non-archimedean field.
    Set \(I_{r,<} := B(0,r)\) and \(I_{r,\leq} := E(0,r)\) for each \(r \in [0,1]\).

    \begin{proposition}\label{prop:ideal_of_integers_ring_of_NA_field}
        The sets \(I_{r,<}\) and \(I_{r,\leq}\) are ideals of the ring of integers \(\kk^\circ\).
        Conversely, any ideal of \(\kk^\circ\) is of the form \(I_{r,<}\) or \(I_{r,\leq}\) for some \(r \in (0,1)\).
        % \Yang{To be checked.}
    \end{proposition}
    \begin{proof}
        Let \(I\) be an ideal of \(\kk^\circ\).
        Set \(r = \sup\{ |a| : a \in I \} \) (resp. \(r= \max\{ |a| : a \in I \} \) when the maximum exists).
        Then, by definition, we have \(I \subset I_{r,<}\) (resp. \(I \subset I_{r,\leq}\)).
        For every \(x \in \kk^\circ\) with \(|x| < r\) (resp. \(|x| \leq r\)), there exists \(a \in I\) such that \(|x| \leq |a|\).
        Thus, \(|x/a| \leq 1\) and so \(x/a \in \kk^\circ\).
        Since \(I\) is an ideal, we have \(x = (x/a) a \in I\).
        Therefore, \(I_{r,<} \subset I\) (resp. \(I_{r,\leq} \subset I\)).
    \end{proof}

    \begin{proposition}\label{prop:recover_complete_non-archimedean_fields_from_projective_limits}
        Let \(I_r\) be either \(I_{r,<}\) or \(I_{r,\leq}\) for each \(r \in (0,1)\).
        Suppose \(\{r_n \in (0,1)\}_{n \in \bbN }\) is a decreasing sequence converging to \(0\).
        Then the completion \(\widehat{\kk}\) of \(\kk\) is isomorphic to the projective limit
        \[ \widehat{\kk}^\circ \cong \varprojlim_{n \in \bbN} \kk^\circ / I_{r_n}. \]
        % \Yang{To be checked.}
    \end{proposition}
    \begin{proof}
        For every \(x \in \widehat{\kk}^\circ\), there exists a cauchy sequence \(\{x_m\}_{m \in \bbN}\) in \(\kk^\circ\) converging to \(x\).
        Since \(\{r_n\}_{n \in \bbN}\) converges to \(0\), for each \(n \in \bbN\), there exists \(M_n \in \bbN\) such that for all \(m, m' \geq M_n\), we have \(|x_m - x_{m'}| < r_n\).
        Thus, the sequence \(\{x_m + I_{r_n}\}_{m \in \bbN}\) is eventually constant in \(\kk^\circ / I_{r_n}\).
        Define a map
        \[
            \Phi : \widehat{\kk}^\circ \to \varprojlim_{n \in \bbN} \kk^\circ / I_{r_n}, \quad x \mapsto \left(\lim_{m\to \infty} x_m + I_{r_n}\right)_{n \in \bbN}.
        \]
        It is straightforward to verify that \(\Phi\) is a well-defined ring homomorphism.

        Conversely, for every \((a_n + I_{r_n})_{n \in \bbN} \in \varprojlim_{n \in \bbN} \kk^\circ / I_{r_n}\), we can choose a representative \(a_n \in \kk^\circ\) for each \(n\).
        We claim that the sequence \(\{a_n\}_{n \in \bbN}\) is a cauchy sequence in \(\kk^\circ\).
        Indeed, for every \(\varepsilon > 0\), there exists \(N \in \bbN\) such that \(r_N < \varepsilon\).
        For all \(m, n \geq N\), since \(a_n + I_{r_n}\) maps to \(a_m + I_{r_m}\) under the natural projection, we have \(|a_n - a_m| < r_N < \varepsilon\).
        Thus, \(\{a_n\}_{n \in \bbN}\) converges to some \(x \in \widehat{\kk}^\circ\).
        Easily see that the limit \(x\) is independent of the choice of representatives \(\{a_n\}_{n \in \bbN}\).
        This gives a map
        \[
            \Psi : \varprojlim_{n \in \bbN} \kk^\circ / I_{r_n} \to \widehat{\kk}^\circ, \quad (a_n + I_{r_n})_{n \in \bbN} \mapsto \lim_{n\to \infty} a_n.
        \]
        Direct verification shows that \(\Psi = \Phi^{-1}\).
    \end{proof}

    \begin{corollary}\label{prop:residue_field_of_completion_is_equal_to_the_original_fields}
        Let \(\kk\) be a non-archimedean field and \(\widehat{\kk}\) its completion.
        Then the residue field \(\calk_{\widehat{\kk}} \cong \calk_\kk\) under the natural embedding \(\kk^\circ \hookrightarrow \widehat{\kk}^\circ\).
    \end{corollary}

    \begin{corollary}\label{prop:valuation_group_of_completion_is_equal_to_the_original_fields}
        Let \(\kk\) be a non-archimedean field and \(\widehat{\kk}\) its completion.
        Then the valuation group \(|\widehat{\kk}^\times|\) of \(\widehat{\kk}\) is equal to the valuation group \(|\kk^\times|\) of \(\kk\).
    \end{corollary}
    \begin{proof}
        Note that 
        \begin{align*}
            r \in |\widehat{\kk}^\times| &\iff I_{r,<} \subsetneqq I_{r,\leq} \text{ in } \widehat{\kk}^\circ \\
            &\iff \widehat{\kk}^\circ / I_{r,<} \to \widehat{\kk}^\circ / I_{r,\leq} \text{ is not an isomorphism} \\
            &\iff \kk^\circ / I_{r,<} \to \kk^\circ / I_{r,\leq} \text{ is not an isomorphism} \\
            &\iff I_{r,<} \subsetneqq I_{r,\leq} \text{ in } \kk^\circ \\
            &\iff r \in |\kk^\times|.
        \end{align*}
    \end{proof}

    \begin{proposition}\label{prop:locally_compact_NA_field_iff_it_is_pro-finite}
        Let \(\kk\) be a non-archimedean field with non-trivial valuation.
        Then \(\kk^\circ\) is totally bounded iff \(\kk^\circ / I_{r,<}\) and \(\kk^\circ / I_{r,\leq}\) are finite for each \(r \in [0,1]\).
        Moreover, if \(\kk\) is complete, then it is locally compact iff \(\kk^\circ/I_r\) is finite for each \(r \in (0,1)\).
        % \Yang{To be checked.}
    \end{proposition}
    \begin{slogan}
        ``Locally compact \(\iff\) pro-finite.''
    \end{slogan}
    \begin{proof}
        We just prove the case for \(I_r = I_{r,<}\). 
        The case for \(I_r = I_{r,\leq}\) is similar.

        Suppose that \(\kk^\circ / I_r\) is finite for each \(r \in [0,1]\).
        Then for every \(\varepsilon > 0\), there exists \(r \in (0,1)\) such that \(r < \varepsilon\) and \(\kk^\circ / I_r\) is finite.
        Let \(\{a_1 + I_r, \ldots, a_n + I_r\}\) be the complete set of representatives of \(\kk^\circ / I_r\).
        Then the balls \(B(a_i,r)\) for \(i=1,\ldots,n\) cover \(\kk^\circ\).

        Conversely, suppose that \(\kk^\circ/I_r\) is infinite for some \(r \in [0,1]\).
        Then there exists an infinite set \(\{a_n\}\) with \(|a_n| \in [r,1]\) such that their images in \(\kk^\circ / I_r\) are distinct.
        In particular, for every \(m \neq n\), we have \(|a_n - a_m| \geq r\).
        Any subsequence of \(\{a_n\}\) is not cauchy.
        Thus, \(\kk^\circ\) is not totally bounded.
    \end{proof}

    \begin{proposition}\label{prop:integers_ring_of_NA_field_is_noetherian_iff_discrete_valuation}
        The ring \(\kk^\circ\) is noetherian iff \(\kk\) is a discrete valuation field.
        % \Yang{To be revised.}
    \end{proposition}
    \begin{proof}
        Note that \(|\kk^\times| \subset \bbR_{>0}\) is a multiplicative subgroup.
        If \(\kk\) is not a discrete valuation field, then \(|\kk^\times|\) is dense in \(\bbR_{>0}\).
        In particular, there exists a strictly ascending sequence \(r_n \in |\kk^\times| \cap (0,1)\).
        Then the ideals \(I_{r_n,\leq}\) form a strictly ascending chain of ideals in \(\kk^\circ\).

        The converse is standard since now \(\kk^\circ\) is a discrete valuation ring.
    \end{proof}

    \begin{proposition}\label{prop:locally_compact_NA_fields_iff_finite_residue_field_and_discrete_valuation}
        Let \(\kk\) be a complete non-archimedean field. 
        Then \(\kk\) is locally compact iff \(\kk\) is a discrete valuation field and its residue field \(\calk_\kk\) is finite.
        % \Yang{To be checked.}
    \end{proposition}
    \begin{proof}
        The necessity follows from \cref{prop:locally_compact_NA_field_iff_it_is_pro-finite}.
        For the sufficiency, suppose that \(\kk\) is a discrete valuation field whose residue field \(\calk_\kk\) is finite.
        Let \(\pi \in \kk^\circ\) be a uniformizer.
        We only need to show that \(\kk^\circ/ \pi^n \kk^\circ\) is finite for each \(n \in \bbN\).
        Note that there is an isomorphism
        \[ \pi^{n-1} \kk^\circ / \pi^n \kk^\circ \cong \calk_\kk, \quad x + \pi^n \kk^\circ \mapsto \overline{x/\pi^{n-1}}. \]
        Thus, by induction on \(n\), we conclude that \(\kk^\circ / \pi^n \kk^\circ\) is finite.
        % \Yang{To be added.}
    \end{proof}



\subsection{Hensel's Lemma}

\begin{theorem}[Hensel's lemma]\label{prop:Hensel_lemma}
        Let \(\kk\) be a complete non-archimedean field and \(F(T) \in \kk^\circ[T]\) a monic polynomial.
        Suppose that the reduction \(f(T) \in \calk_\kk[T]\) of \(F(T)\) factors as
        \[ f(T) = g(T) h(T), \]
        where \(g(T), h(T) \in \calk_\kk[T]\) are monic polynomials that are coprime in \(\calk_\kk[T]\).
        Then there exist monic polynomials \(G(T), H(T) \in \kk^\circ[T]\) such that
        \[ F(T) = G(T) H(T), \]
        and the reductions of \(G(T), H(T)\) in \(\calk_\kk[T]\) are \(g(T), h(T)\) respectively.
        % \Yang{To be checked.}
    \end{theorem}
    \begin{proof}
        Since \(\gcd(g,h) = 1\) in \(\calk_\kk[T]\), there exist polynomials \(u(T), v(T) \in \calk_\kk[T]\) such that \(ug + vh = 1\) and \(\deg u < \deg h, \deg v < \deg g\).
        Choose lifts \(G_0(T), H_0(T), U(T), V(T) \in \kk^\circ[T]\) of \(g(T), h(T), u(T), v(T)\) respectively preserving their degrees such that \(G_0\) and \(H_0\) are monic.
        Then there exist \(r<1\) such that 
        \[ U(T) G_0(T) + V(T) H_0(T) \equiv 1 \mod I_{r},\quad F(T) - G_0(T) H_0(T) \equiv 0 \mod I_{r}, \]
        where \(I_{r} = \{ a \in \kk^\circ : |a| < r \}\).

        We will construct a sequence of monic polynomials \(\{G_n(T)\}_{n \in \bbN}\) and \(\{H_n(T)\}_{n \in \bbN}\) in \(\kk^\circ[T]\) such that for each \(n \in \bbN\),
        \[ G_n(T) \equiv G_{n-1}(T) \mod I_{r^n}, \quad H_n(T) \equiv H_{n-1}(T) \mod I_{r^n}, \]
        and
        \[ F(T) - G_n(T) H_n(T) \equiv 0 \mod I_{r^{n+1}}. \]
        If we have such sequences, then their coefficients converge in the complete ring \(\kk^\circ\).
        Let \(G(T)\) and \(H(T)\) be the limits of \(\{G_n(T)\}\) and \(\{H_n(T)\}\) respectively.
        Then we have \(F(T) = G(T) H(T)\) and the reductions of \(G(T), H(T)\) in \(\calk_\kk[T]\) are \(g(T), h(T)\) respectively.

        The case \(n=0\) is done by the above construction.
        Now suppose that we have constructed \(G_n(T)\) and \(H_n(T)\) for some \(n \geq 0\).
        Since \(G_n - G_0 \equiv 0 \mod I_{r}\) and \(H_n - H_0 \equiv 0 \mod I_{r}\), we have
        \[ U G_n + V H_n = U G_0 + V H_0 + U (G_n - G_0) + V (H_n - H_0) \equiv 1 \mod I_{r}. \]
        Set \(\Delta_n(T) = F(T) - G_n(T) H_n(T) \in I_{r^{n+1}}[T]\) and \(\epsilon_n = U \Delta_n, \delta_n = V \Delta_n \in I_{r^{n+1}}[T]\).
        Then we have 
        \begin{align*}
            (G_n + \epsilon_n)(H_n + \delta_n) - F_n &= G_n H_n + G_n \delta_n + H_n \epsilon_n + \epsilon_n \delta_n - F_n \\
            &= (U G_n + V H_n - 1) \Delta_n + \epsilon_n \delta_n \in I_{r^{n+2}}[T].
        \end{align*}
        Thus, we can set
        \[ G_{n+1}(T) = G_n(T) + \epsilon_n(T), \quad H_{n+1}(T) = H_n(T) + \delta_n(T). \]
        This finishes the induction.
        % \Yang{To be added.}
    \end{proof}

    \begin{corollary}\label{prop:Hensel_lemma_to_solve_polynomial}
        Let \(\kk\) be a complete non-archimedean field and \(F(T) \in \kk^\circ[T]\) a monic polynomial.
        Suppose that the reduction \(f(T) \in \calk_\kk[T]\) of \(F(T)\) has a simple root \(a \in \calk_\kk\).
        Then there exists a root \(\alpha \in \kk^\circ\) of \(F(T)\) whose reduction is \(a\).
        % \Yang{To be revised.}
    \end{corollary}
    \begin{proof}
        Since \(a\) is a simple root of \(f(T)\), we have the factorization \(f(T) = (T - a) h(T)\) for some monic polynomial \(h(T) \in \calk_\kk[T]\) with \(h(a) \neq 0\).
        Then the result follows from \cref{prop:Hensel_lemma}.
        % \Yang{To be added.}
    \end{proof}


\subsection{Newton polygons}

    \Yang{To be filled.}
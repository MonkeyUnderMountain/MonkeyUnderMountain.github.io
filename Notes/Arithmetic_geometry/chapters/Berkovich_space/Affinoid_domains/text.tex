\section{Affinoid domains}

    Consider \(X = \scrM(A)\) with \(A = \kk\{T_1,\ldots,T_n\}\).
    \Yang{Not every open subset of \(X\) gives an affinoid space, that is, the completion of the ring of analytic functions on that open subset is not necessarily an affinoid algebra.}
    \Yang{Right? example?}

\subsection{Definition}


    \begin{definition}\label{def:affinoid_domains}
        Let \(A\) be a \(\kk\)-affinoid algebra, and let \(X = \scrM(A)\) be the associated affinoid space.
        A closed subset \(V \subseteq X\) is called an \emph{affinoid domain} if there exists a \(\kk\)-affinoid algebra \(A_V\) and a morphism of \(\kk\)-affinoid algebras \(\varphi: A \to A_V\) satisfying the following universal property:
        for every bounded homomorphism of \(\kk\)-affinoid algebras \(\psi: A \to B\) such that the induced map on spectra \(\scrM(\psi): \scrM(B) \to X\) has its image contained in \(V\), there exists a unique bounded homomorphism \(\theta: A_V \to B\) such that the following diagram commutes:
        \[
        \begin{tikzcd}
            & A_V  \arrow[dr, gray, "\theta"] & \\
            A \arrow[ur, "\varphi"] \arrow[rr, "\psi"'] & & B
        \end{tikzcd}
        \]
        In this case, we say that \(V\) is represented by the affinoid algebra \(A_V\).
    \end{definition}
    \begin{slogan}
        A closed subset \(V \subset X\) is an affinoid domain if the functor ``\(\Mor(-, V)\)'' is representable.
    \end{slogan}

    \Yang{Why we consider closed subset rather that open subset?}


    \begin{construction}\label{constr:weierstrass_domain}
        Let \(f=(f_1,\ldots,f_n)\) be a tuple of elements in \(A\) and \(r=(r_1,\ldots,r_n)\) be a tuple of positive real numbers.
        Consider the closed subset of \(X\):
        \[ X\left(\underline{f/r}\right) := \left\{ x \in X \colon |f_i(x)| \le r_i, 1 \le i \le n \right\}. \]
        Such a closed subset is called a \emph{Weierstrass domain} of \(X\).
        Moreover, we can define a \(\kk\)-affinoid algebra
        \[ A\left\{ \underline{f/r} \right\} := A\left\{ f_1/r_1,\ldots,f_n/r_n \right\}. \]
        \Yang{The domain \(X\left(\underline{f/r}\right)\) is represented by \(A\left\{\underline{f/r}\right\}\).}
    \end{construction}

    \begin{construction}\label{constr:Laurent_domain}
        Let \(f=(f_1,\ldots,f_n),g = (g_1,\ldots,g_m)\) be two tuples of elements in \(A\) and \(r=(r_1,\ldots,r_n),s=(s_1,\ldots,s_m)\) be two tuples of positive real numbers.
        Consider the following closed subset of \(X\):
        \[ X\left(\underline{f/r};\underline{g/s}^{-1}\right) := \left\{ x \in X \colon |f_i(x)| \le r_i, |g_j(x)| \ge s_j, 1 \le i \le n, 1 \le j \le m \right\}. \]
        Such a closed subset is called a \emph{Laurent domain} of \(X\).
        Moreover, we can define a \(\kk\)-affinoid algebra
        \[ A\left\{ \underline{f/r};\underline{g/s}^{-1} \right\} := A\left\{ f_1/r_1,\ldots,f_n/r_n,g_1^{-1}/s_1,\ldots,g_m^{-1}/s_m \right\}. \]
        \Yang{The domain \(X\left(\underline{f/r};\underline{g/s}^{-1}\right)\) is represented by \(A\left\{\underline{f/r};\underline{g/s}^{-1}\right\}\).}
    \end{construction}

    \begin{construction}\label{constr:rational_domain}
        Let \(f=(f_1,\ldots,f_n),g\) be elements in \(A\) such that the ideal generated by them is the whole algebra \(A\).
        Set \(p=(p_1,\ldots,p_n)\) be a tuple of positive real numbers.
        We define the following closed subset of \(X\):
        \[ X\left(\underline{f/p},g\right) := \left\{ x \in X \colon |f_i(x)| \le p_i |g(x)|, 1 \le i \le n \right\}. \]
        Such a closed subset is called a \emph{rational domain} of \(X\).
        Moreover, we can define a \(\kk\)-affinoid algebra
        \[ A\left\langle \underline{f/p},g^{-1} \right\rangle := A\left\langle \frac{f_1}{p_1 g},\ldots,\frac{f_n}{p_n g} \right\rangle, \]
        which is the quotient of the Tate algebra
        \[ A\left\langle T_1,\ldots,T_n \right\rangle \]
        by the ideal generated by the elements \(p_i g T_i - f_i\) for \(1 \le i \le n\).
        There is a natural bounded homomorphism \(\varphi: A \to A\langle \underline{f/p},g^{-1} \rangle\) induced by the inclusion.
        It can be shown that the closed subset \(X(\underline{f/p},g)\) is an affinoid domain represented by the affinoid algebra \(A\langle \underline{f/p},g^{-1} \rangle\).
        \Yang{To be checked}
    \end{construction}

    \Yang{We have a sequence of inclusion:}
    \[ \{\text{Weierstrass domains}\} \subseteq \{\text{Laurent domains}\} \subseteq \{\text{Rational domains}\} \subseteq \{\text{Affinoid domains}\}. \]

    \begin{proposition}\label{prop:affinoid_domain_is_flat_over_base}
        Let \(A\) be a \(\kk\)-affinoid algebra, and let \(X = \scrM(A)\) be the associated affinoid space.
        Let \(V \subseteq X\) be an affinoid domain represented by the \(\kk\)-affinoid algebra \(A_V\).
        Then the natural bounded homomorphism \(\varphi: A \to A_V\) is flat.

        We have \(\scrM(A_V) \cong V\).
    \end{proposition}


\subsection{The Grothendieck topology of affinoid domains}

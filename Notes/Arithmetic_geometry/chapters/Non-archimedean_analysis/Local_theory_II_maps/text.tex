\section{Local theory II: maps}

Let \(\kk\) be a complete non-archimedean field.

\subsection{The first properties}

    \Yang{Recall the Runge theorem in complex analysis.}

    \begin{definition}\label{def:analytic_function_on_closed_subset_of_k^n}
        A map \(f:(E(0,r) \subset \kk^n) \to \kk^m\) is called \emph{analytic} 
        if there exists power series \(f_1,\ldots,f_m \in \kk\{\underline{T/r}\}\) 
        such that for any \(x \in E(0,r)\), we have
        \[f(x) = (f_1(x),\ldots,f_m(x)).\]
        \Yang{To be revised.}
    \end{definition}

    \Yang{Composition of analytic functions.}

    \begin{definition}\label{def:analytic_map_local_version}
        A map \(f:(E(0,\underline{r}) \subset \kk^n) \to \kk^m\) is called \emph{analytic} 
        if there exists power series \(f_1,\ldots,f_m \in \kk\{\underline{T/r}\}\) 
        such that for any \(x \in E(0,\underline{r})\), we have
        \[f(x) = (f_1(x),\ldots,f_m(x)).\]
    \end{definition}

    \begin{proposition}\label{prop:composition_of_analytic_map}
        Let \(f:(E(0,\underline{r}) \subset \kk^n) \to \kk^m\) and 
        \(g:(E(0,\underline{s}) \subset \kk^m) \to \kk^l\) be two analytic maps
        such that \(f(E(0,\underline{r})) \subset E(0,\underline{s})\).
        Then the composition \(g \circ f:(E(0,\underline{r}) \subset \kk^n) \to \kk^l\) is also analytic.
        
        Furthermore, if \(f = (f_1,\ldots,f_m)\) and \(g = (g_1,\ldots,g_l)\) with \(f_i = \sum_{\alpha} a_{i,\alpha} \underline{T}^\alpha\) and \(g_j = \sum_{\beta} b_{j,\beta} \underline{T}^\beta\),
        then the composition \(g \circ f = (h_1,\ldots,h_l)\) with
        \[h_j = \sum_{\beta} b_{j,\beta} f^{\beta} = \sum_{\beta} b_{j,\beta} f_1^{\beta_1} \cdots f_m^{\beta_m}.\]
        \Yang{To be checked.}
        \Yang{To be revised.}
    \end{proposition}
    \begin{proof}
        \Yang{To be completed.}
    \end{proof}


\subsection{Inverse and implicit function}

    \begin{definition}\label{def:tangent_map_of_an_analytic_map}
        Let \(f:(E(0,\underline{r}) \subset \kk^n) \to \kk^m\) be an analytic map.
        The \emph{tangent map} \(\upd f_0:\kk^n \to \kk^m\) of \(f\) at \(0\) is defined to be the linear map
        given by the Jacobian matrix
        \[\left(\frac{\partial f_i}{\partial T_j}(0)\right)_{1 \leq i \leq m, 1 \leq j \leq n}.\]
        \Yang{To be checked.}
    \end{definition}

    \begin{theorem}[Inverse Function Theorem over Non-Archimedean Fields]\label{thm:inverse_function_theorem_over_NA_fields}
        Let \(f:(E(0,\underline{r}) \subset \kk^n) \to \kk^n\) be an analytic map.
        Suppose that \(f(0) = 0\) and the tangent map \(\upd f_0:\kk^n \to \kk^n\) is an isomorphism.
        
        Then there exist \(E(0,\underline{r'}) \subset E(0,\underline{r})\), \(E(0,\underline{s'}) \subset f(E(0,\underline{r}))\) and an analytic map
        \(g:(E(0,\underline{s'}) \subset \kk^n) \to \kk^n\)
        such that 
        \[f \circ g = \id_{E(0,\underline{s'})}, \quad g \circ f = \id_{E(0,\underline{r'})}.\]
    \end{theorem}
    \begin{proof}
        \Yang{To be completed.}
    \end{proof}

    \begin{theorem}[Implicit Function Theorem over Non-Archimedean Fields]\label{thm:implicit_function_theorem_over_NA_fields}
        Let \(f:(E(0,\underline{r}) \subset \kk^{n+m}) \to \kk^m, (x_1,\cdots,x_n,y_1,\ldots,y_m) \mapsto f(x,y)\) be an analytic map.
        Suppose that \(f(0) = 0\) and the Jacobian matrix \(\left(\partial_j f_i(0)\right)_{1 \leq i,j \leq m}\) is invertible.
        
        Then there exist \(\underline{r'} = (r_1',\ldots,r_n')\) with each \(r_i' > 0\)
        and an analytic map \(g:(E(0,\underline{r'}) \subset \kk^n) \to \kk^m\)
        such that for any \(x \in E(0,\underline{r'})\),
        \[f(x, y) = 0 \iff y = g(x).\]
    \end{theorem}
    \begin{proof}
        \Yang{To be completed.}
    \end{proof}